%%%%%%%%%%%%%%%%%%%%%%%%%%%%%%%%%%%%%%%%%
% Journal Article
% LaTeX Template
% Version 1.3 (9/9/13)
%
% This template has been downloaded from:
% http://www.LaTeXTemplates.com
%
% Original author:
% Frits Wenneker (http://www.howtotex.com)
%
% License:
% CC BY-NC-SA 3.0 (http://creativecommons.org/licenses/by-nc-sa/3.0/)
%
%%%%%%%%%%%%%%%%%%%%%%%%%%%%%%%%%%%%%%%%%

%----------------------------------------------------------------------------------------
%	PACKAGES AND OTHER DOCUMENT CONFIGURATIONS
%----------------------------------------------------------------------------------------

\documentclass[twoside]{article}

\usepackage{lipsum} % Package to generate dummy text throughout this template

\usepackage[sc]{mathpazo} % Use the Palatino font
\usepackage[T1]{fontenc} % Use 8-bit encoding that has 256 glyphs
\linespread{1.05} % Line spacing - Palatino needs more space between lines
\usepackage{microtype} % Slightly tweak font spacing for aesthetics

\usepackage[hmarginratio=1:1,top=32mm,columnsep=20pt]{geometry} % Document margins
\usepackage{multicol} % Used for the two-column layout of the document
\usepackage[hang, small,labelfont=bf,up,textfont=it,up]{caption} % Custom captions under/above floats in tables or figures
\usepackage{booktabs} % Horizontal rules in tables
\usepackage{float} % Required for tables and figures in the multi-column environment - they need to be placed in specific locations with the [H] (e.g. \begin{table}[H])
\usepackage{hyperref} % For hyperlinks in the PDF

\usepackage{lettrine} % The lettrine is the first enlarged letter at the beginning of the text
\usepackage{paralist} % Used for the compactitem environment which makes bullet points with less space between them

\usepackage{abstract} % Allows abstract customization
\renewcommand{\abstractnamefont}{\normalfont\bfseries} % Set the "Abstract" text to bold
\renewcommand{\abstracttextfont}{\normalfont\small\itshape} % Set the abstract itself to small italic text

\usepackage{titlesec} % Allows customization of titles
\renewcommand\thesection{\Roman{section}} % Roman numerals for the sections
\renewcommand\thesubsection{\Roman{subsection}} % Roman numerals for subsections
\titleformat{\section}[block]{\large\scshape\centering}{\thesection.}{1em}{} % Change the look of the section titles
\titleformat{\subsection}[block]{\large}{\thesubsection.}{1em}{} % Change the look of the section titles

\usepackage{fancyhdr} % Headers and footers
\pagestyle{fancy} % All pages have headers and footers
\fancyhead{} % Blank out the default header
\fancyfoot{} % Blank out the default footer
\fancyhead[C]{Running title $\bullet$ March 2015 $\bullet$ Vol. XXI, No. 1} % Custom header text
\fancyfoot[RO,LE]{\thepage} % Custom footer text

\usepackage{relsize}

%----------------------------------------------------------------------------------------
%	TITLE SECTION
%----------------------------------------------------------------------------------------

\title{\vspace{-15mm}\fontsize{24pt}{10pt}\selectfont\textbf{Software Bot Taxonomy}} % Article title

\author{
\large
\textsc{Graded Student in Computer Science Axel Alejandro Garcia Fuentes}\thanks{-}\\[2mm] % Your name
\normalsize Universidad Autonoma de Guadalajara \\ % Your institution
\normalsize \href{mailto:axel.garcia@edu.uag.mx}{axel.garcia@edu.uag.mx} % Your email address
\vspace{-5mm}
}
\date{}

%----------------------------------------------------------------------------------------

\begin{document}

\maketitle % Insert title

\thispagestyle{fancy} % All pages have headers and footers

%----------------------------------------------------------------------------------------
%	ABSTRACT
%----------------------------------------------------------------------------------------

\begin{abstract}

%\noindent \lipsum[1] % Dummy abstract text
Abstract
\end{abstract}

%----------------------------------------------------------------------------------------
%	ARTICLE CONTENTS
%----------------------------------------------------------------------------------------

\begin{multicols}{2} % Two-column layout throughout the main article text

\section{Introduction}
\lettrine[nindent=0em,lines=3]{T}his document ...


%------------------------------------------------
\section{Background}
Definition of boot from:
Science Bots: a Model for the Future of Scientific Computation?: http://arxiv.org/pdf/1503.04374.pdf

How good boots are used: Wikipedia Bot Writes 10,000 Articles a Day: http://news.discovery.com/tech/robotics/wikipedia-bot-writes-10000-articles-a-day-140715.htm

%------------------------------------------------
\section{Problem}
Mention section Engineered social tampering of The Rise of Social Bots: http://arxiv.org/pdf/1407.5225v2.pdf
%Reference \cite{saif.ea:2012} mentions 

%------------------------------------------------
\section{Taxonomies}
Discuss taxonomies:
 - Wikipedia:Types of bots: http://en.wikipedia.org/wiki/Wikipedia:Types_of_bots
 - Different Types of Bots: http://www.honeynet.org/node/53
 - Different Types of Internet Bots and How They Are Used: http://www.spamlaws.com/how-internet-bots-are-used.html
 - Official List of Bot Types: http://realsteel.wikia.com/wiki/Official_List_of_Bot_Types
 - Bot types: http://botology.tumblr.com/types

%------------------------------------------------
\section{Detection}
Mention Researchers Release Twitter Bot Detection Tool: http://www.tripwire.com/state-of-security/latest-security-news/researchers-release-twitter-bot-detection-tool/
Include example of: Bot or Not? A Truthy project: http://truthy.indiana.edu/botornot/


%----------------------------------------------------------------------------------------
%	REFERENCE LIST
%----------------------------------------------------------------------------------------
http://arxiv.org/pdf/1503.04374.pdf
http://arxiv.org/find/all/1/all:+Bot/0/1/0/all/0/1
http://arxiv.org/pdf/1407.5225v2.pdf
http://truthy.indiana.edu/botornot/
http://www.tripwire.com/state-of-security/latest-security-news/researchers-release-twitter-bot-detection-tool/
https://www.shadowserver.org/wiki/pmwiki.php/Information/Honeypots

https://www.google.com.mx/?gfe_rd=cr&ei=AZcLVYu7JojUpAOEmYKADA&gws_rd=ssl#q=bot+detection

% Steps for typesetting document with Bibtex:
% 1) Latex
% 2) Bibtex
% 3) Latex
% 4) Latex

\bibliographystyle{apalike} % acm, ieeetr, apalike, ...
\bibliography{biblio}

%----------------------------------------------------------------------------------------

\end{multicols}

\end{document}
